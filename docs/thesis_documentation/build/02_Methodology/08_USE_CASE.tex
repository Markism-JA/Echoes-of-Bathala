\section{Use Cases}\label{use-cases}

\begin{figure}[H]
\centering
\pandocbounded{\includegraphics[width=\textwidth, keepaspectratio]{assets/usecase/main-use-case.pdf}}
\caption{Proposed Main Use Case}
\end{figure}

\begin{figure}[H]
\centering
\pandocbounded{\includegraphics[width=\textwidth, keepaspectratio]{assets/usecase/economy-use-case.pdf}}
\caption{Proposed Economic Use Case}
\end{figure}

\begin{figure}[H]
\centering
\pandocbounded{\includegraphics[width=\textwidth]{assets/usecase/baku-coin-use-case.pdf}}
\caption{Proposed Baku Coin and Perlas Use Case}
\end{figure}

\subsection{Detailed Use Case Description for BAKU and Perlas}

\renewcommand{\arraystretch}{1.3}

% UC-1
\begin{table}[H]
\centering
\begin{tabularx}{\textwidth}{|l|X|}
\hline
\textbf{Use Case ID} & \textbf{UC-1: Acquire Off-Chain Resource (Perlas)} \\ \hline
\textbf{Actor} & Player \\ \hline
\textbf{Description} & The Player performs in-game activities to earn Perlas. This is the primary ``faucet'' for the off-chain economy. \\ \hline
\textbf{Goal} & To reward the Player's time and effort with a liquid in-game currency. \\ \hline
\textbf{Main Flow} & 
\begin{enumerate}[nosep, leftmargin=*, after=\vspace{\baselineskip}]
    \item The Player completes a ``faucet'' activity (e.g., defeats a monster, completes a quest).
    \item The system grants Perlas and adds it to the Player's off-chain inventory (Game Database).
\end{enumerate} \\ \hline
\textbf{Postcondition} & The Player's Perlas balance is increased. \\ \hline
\end{tabularx}
\end{table}

% UC-2
\begin{table}[H]
\centering
\begin{tabularx}{\textwidth}{|l|X|}
\hline
\textbf{Use Case ID} & \textbf{UC-2: Spend Off-Chain Resource (Perlas Sinks)} \\ \hline
\textbf{Actor} & Player \\ \hline
\textbf{Description} & The Player spends Perlas on cyclical in-game services and taxes. \\ \hline
\textbf{Goal} & To create a constant, low-level drain on the Perlas supply. \\ \hline
\textbf{Main Flow} & \textit{(Example: Item Repair)}
\begin{enumerate}[nosep, leftmargin=*, after=\vspace{\baselineskip}]
    \item The Player's equipment (both normal and tokenized) has low durability.
    \item The Player interacts with a Blacksmith and selects ``Repair.''
    \item The system displays the total Perlas cost for the repair.
    \item The Player confirms. The system subtracts the Perlas.
    \item The system restores the item's durability.
\end{enumerate} \\ \hline
\textbf{Alt. Flows} & \textit{(Other Sinks)}
\begin{itemize}[nosep, leftmargin=*]
    \item \textbf{Talipapa Tax:} A 5\% Perlas fee on all marketplace sales.
    \item \textbf{Fast Travel:} Perlas fees for convenient travel.
    \item \textbf{Consumables:} Perlas cost for potions, food, etc.
\end{itemize} \\ \hline
\end{tabularx}
\end{table}

% UC-3
\begin{table}[H]
\centering
\begin{tabularx}{\textwidth}{|l|X|}
\hline
\textbf{Use Case ID} & \textbf{UC-3: Forge On-Chain Token (BAKU) - (The Faucet)} \\ \hline
\textbf{Actor} & Player \\ \hline
\textbf{Description} & This is the Baku Forge System (3.7). The Player converts their earned Perlas (labor) into the on-chain BAKU token (capital). This is the only source of new BAKU. \\ \hline
\textbf{Goal} & To provide the single, player-driven ``faucet'' that creates BAKU. \\ \hline
\textbf{Preconditions} & 
\begin{itemize}[nosep, leftmargin=*]
    \item The Player has a linked and verified external wallet.
    \item The Player has a sufficient Perlas balance.
\end{itemize} \\ \hline
\textbf{Main Flow} & 
\begin{enumerate}[nosep, leftmargin=*, after=\vspace{\baselineskip}]
    \item The Player interacts with the ``Baku Forge'' game object.
    \item The UI displays the conversion rate (e.g., ``1000 Perlas = 1 BAKU'').
    \item The Player inputs the amount of BAKU to forge and confirms.
    \item The system destroys the Perlas from the Player's in-game inventory.
    \item The Game Server sends a request to the Token Infrastructure to transfer the BAKU to the Player's linked wallet.
\end{enumerate} \\ \hline
\textbf{Postcondition} & The Player's Perlas is permanently decreased. The Player's BAKU balance (on-chain) is increased. \\ \hline
\end{tabularx}
\end{table}

% UC-4
\begin{table}[H]
\centering
\begin{tabularx}{\textwidth}{|l|X|}
\hline
\textbf{Use Case ID} & \textbf{UC-4: Spend BAKU on In-Game Progression} \\ \hline
\textbf{Actor} & Player \\ \hline
\textbf{Description} & The Player spends BAKU as a required material or service fee for high-stakes, in-game progression. \\ \hline
\textbf{Goal} & To create a powerful in-game utility (a ``sink'') for BAKU. \\ \hline
\textbf{Sub-Flow 1} & \textit{(Crafting Legendary Item)}
\begin{enumerate}[nosep, leftmargin=*, after=\vspace{\baselineskip}]
    \item The Player has the rare (non-token) materials to craft a Legendary Item.
    \item The recipe also requires a number of BAKU as a core material.
    \item The Player confirms the crafting attempt.
    \item The system verifies all materials and the Player's BAKU balance.
    \item The system consumes in-game materials and sends a ``consume'' request to the Token Infrastructure for BAKU (transferring it to a burn wallet).
    \item The Legendary Item is created and placed in the Player's inventory.
\end{enumerate} \\ \hline
\textbf{Sub-Flow 2} & \textit{(High-Stakes Enhancement)}
\begin{enumerate}[nosep, leftmargin=*, after=\vspace{\baselineskip}]
    \item The Player attempts to enhance an item from +8 to +9.
    \item The Enhancement UI presents a checkbox: ``Use Blessing (Cost: 1 BAKU)'' to prevent destruction on failure.
    \item The Player checks the box and clicks ``Enhance''.
    \begin{itemize}[nosep, label=-]
        \item \textbf{On Failure:} The 1 BAKU is consumed. Item not destroyed.
        \item \textbf{On Success:} The 1 BAKU is not consumed. Item becomes +9.
    \end{itemize}
\end{enumerate} \\ \hline
\textbf{Exception} & \textbf{Insufficient Funds:} If the Player does not have enough BAKU, the ``Craft'' or ``Blessing'' option is disabled. \\ \hline
\end{tabularx}
\end{table}

% UC-5
\begin{table}[H]
\centering
\begin{tabularx}{\textwidth}{|l|X|}
\hline
\textbf{Use Case ID} & \textbf{UC-5: Spend BAKU on On-Chain Asset Minting} \\ \hline
\textbf{Actor} & Player \\ \hline
\textbf{Description} & The Player spends BAKU as a one-time fee to make their in-game item or character externally tradable (NFT Minting System 3.8). \\ \hline
\textbf{Goal} & To create an on-chain utility (a ``sink'') for BAKU and manage the creation of NFTs. \\ \hline
\textbf{Precondition} & The Player has an eligible in-game item (e.g., a Legendary Item). \\ \hline
\textbf{Main Flow} & \textit{(Mint Item to NFT)}
\begin{enumerate}[nosep, leftmargin=*, after=\vspace{\baselineskip}]
    \item The Player selects the ``Mint to NFT'' option for that item.
    \item The UI displays a confirmation: ``Minting this item will make it tradable on-chain. This action costs 10 BAKU.''
    \item The Player confirms.
    \item The system consumes 10 BAKU.
    \item The Game Server sends the item's metadata to the Token Infrastructure.
    \item An NFT is minted to the Player's wallet.
    \item The in-game item is ``locked'' and flagged as ``tokenized'' in the Game Database.
\end{enumerate} \\ \hline
\textbf{Postcondition} & The Player's BAKU is consumed, and the item is now represented by an on-chain NFT. \\ \hline
\end{tabularx}
\end{table}
