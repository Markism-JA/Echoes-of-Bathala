\section{Human Resources}\label{human-resources}

\subsection{Community Transcript}\label{community-transcript}

\textbf{Compiled Source Threads:}

\begin{enumerate}
\def\labelenumi{\arabic{enumi}.}
\tightlist
\item
  r/CryptoCurrency: ``So what happened to all those NFT games wayback?''
\item
  r/Buttcoin: ``The thing that NFT games and companies seem to\ldots{}''
\item
  r/playtoearngames: ``I've come across a lot of web3 games in the
  past\ldots{}''
\item
  r/Entrepreneur: ``Why haven't I seen any good NFT games?''
\item
  r/web3: ``Gas fees are killing me! Is there any way around?''
\end{enumerate}

\subsubsection{Category 1: Economic Instability Caused by Unbalanced
Tokenomics}\label{category-1-economic-instability-caused-by-unbalanced-tokenomics}

User A (r/CryptoCurrency)

\begin{quote}
``In almost every case: unsustainable tokenomics---all relied on a
constantly growing audience in order to off-set the deflating
asset/token prices. Players were mostly there to extract value, and did
so from each other---but without newcomers there is no one to sell
to/extract from.''
\end{quote}

User B (r/CryptoCurrency)

\begin{quote}
``Most had neither a growth or business model. Nearly all of them relied
on some ponzinomics and wealth effect trickled down from majors to keep
things running. For Axie, I believe the Ponzi was via its SLP token
emission. When all the ponzinomics dried up and retail either cashed out
of the majors or lost their new found wealth on these Ponzi, the
interest on these games also died out.''
\end{quote}

User C (r/playtoearngames)

\begin{quote}
``Sustainability: Many Web3 games like Gods Unchained and Axie Infinity
have shown potential, but sustainability is a major concern.''
\end{quote}

\subsubsection{Category 2: Weak Player Engagement Due to Overemphasis on
Profit-Driven
Mechanics}\label{category-2-weak-player-engagement-due-to-overemphasis-on-profit-driven-mechanics}

User D (r/Entrepreneur)

\begin{quote}
``I don't think any of them worked bc the actual games were boring and
the returns for playing were too low.''
\end{quote}

User E (r/Buttcoin)

\begin{quote}
``Most of them were just straight up unfun and encouraged toxic
community behavior. The big point of realization is that without abusive
ponzi mechanics and rampant speculation (both unhealthy to a fun,
balanced gameplay experience), the economic value any individual
player's effort contributed to the game is astronomically low. Turns out
putting a price tag on the time spent makes the whole thing unrewarding
if it comes out to \$0.10 an hour of play time.''
\end{quote}

User F (r/Buttcoin)

\begin{quote}
``What they don't seem to realize is that items in video games have
value because people find the games themselves fun. People play games to
have fun, not to invest or earn money.''
\end{quote}

\subsubsection{Category 3: Technical Vulnerabilities and the Need for
Efficient Hybrid
Integration}\label{category-3-technical-vulnerabilities-and-the-need-for-efficient-hybrid-integration}

User G (r/Entrepreneur)

\begin{quote}
``The main problem is that they didn't decentralize the NFTs. If they
charged a fee for making an NFT that is recognized in the game which may
be a sword, armor, gun, magic etc then that would work and not be a
burden because people could show off art. It then makes the cash grab
not that but making new NFTs by users.''
\end{quote}

User H (r/web3)

\begin{quote}
``Gas fees are killing me! Is there any way around this?''
\end{quote}

User I (r/web3)

\begin{quote}
``I have been experiencing the same. I was tasked with testing out
certain platforms, and amount of money I lost in Gas fee is insane.
Another issue is, when you are trying to get some recent tokens or
coins, it very difficult and costly to bridge, swap and find routes to
do the same. And like you rightly pointed out, for small amounts is
unusable.''
\end{quote}

User J (r/playtoearngames)

\begin{quote}
``If done right, it could bridge nostalgic gameplay with the future of
digital ownership.''
\end{quote}

\emph{Source: Compiled from cited Reddit discussions. User identifiers
have been anonymized to alphanumeric codes (User A, User B, etc.) for
academic clarity and privacy.}
