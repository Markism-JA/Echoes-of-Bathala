\chapter{Methodology}\label{methodology}

This section articulates the systematic approach and technical
architecture driving the development of ``Echoes of Bathala Online'',
defining the specific development strategy and lifecycle planning
adopted. It exhibits the technical specification by enumerating the
required hardware and software. Furthermore, the methodology includes
budgetary estimates of the project and to visualize the proposed system,
graphic representation such as a high-level conceptual framework and
data flow diagram illustrates the processes and data movement across the
proposed software architecture.

\begin{figure}[htbp]
\centering
\includegraphics[width=\linewidth]{assets/rapid-application-development.pdf}
\centering
\caption{Rapid Application Development}
\end{figure}

Rapid Application Development (RAD) is an adaptive software development
methodology that that emphasizes rapid prototyping, iterative
development, and substantial user involvement to accelerate the delivery
of a quality, functional software product. As such, its implementation
to the development of Echoes of Bathala Online because of the constraint
on time and resources for the project. By facilitating continuous user
feedback throughout the development process, RAD allows the
identification and refinement of the system requirements minimizing the
risk of costly redesigns. Additionally, the fast-moving development
cycles will enable the project team to implement the core
functionalities in order to deliver a playable software product on time.

\begin{enumerate}
\def\labelenumi{\arabic{enumi}.}
\item
  \textbf{Requirements Planning}

  The phase of the requirements planning provides a baseline direction
  to the project and clears expectations from all interested parties. In
  the context of RAD, this phase emphasizes fast but organized teamwork
  to outline the goals and constraints of the system. The activities
  will focus, in this phase, on detailing out the scope of the project,
  objectives of gameplay, and essential functions. This includes
  finalizing the GDD, detailing the hybrid tokenomics model that
  integrates Perlas and Baku Coin, and specifying the technical
  architecture to interface Unity with the blockchain network by using
  ASP.NET Web APIs.

  In this stage, consultations with advisers, developers, and potential
  end users-for example, players and/or evaluators-are made to ascertain
  key needs and possible risks such as gameplay balance and blockchain
  transaction flows. The goal is to arrive to a mutual understanding of
  the intended actions of the system prior to the design and prototyping
  phases. As stressed, early agreement on requirements reduces
  ambiguity, limits rework, and ensures subsequent phases are driven by
  a clear, validated direction.
\item
  \textbf{User Design}

  The user design phase is the iterative and collaborative heart of RAD.
  In this work, prototyping cycles will be used to gradually refine
  those aspects of the system that will be visible to users. Iterative,
  interactive prototypes of central gameplay systems, such as combat
  mechanics, the UI, and the asset conversion mechanism, will be rapidly
  created, tested, and evaluated.

  The continuous feedback by stakeholders, including potential users and
  academic advisors, will point out usability issues, gameplay
  inconsistencies, and improvements in design. Every iteration will give
  the developers the ability to adjust mechanics, interface layouts and
  interaction flows. The confirmation of initial design choices by
  engaging users, facilitating developer conversations, and refining
  through iteration until the prototype adequately represents the
  desired user experience. Addressing key design issues in this stage
  reduces risk and facilitates a more seamless construction process
  construction.
\item
  \textbf{Construction}

  The construction phase is the transformation of the validated
  prototypes into a functional build of Minimum Playable Content.
  Consistent with RAD principles, this stage goes fast because all the
  foundational design issues have been handled in the iterative User
  Design phase. Development activities will be executed in parallel,
  enabling faster advancement without quality compromise.

  Implementation activities will involve the creation of assets,
  gameplay programming using Unity, and server-side development using
  ASP.NET. Integration with blockchain will be implemented along with
  deployment and testing of BAKU (ERC-20) and NFT (ERC-721) smart
  contracts on Polygon testnet using Hardhat and Nethereum. Frequent
  testing and building incrementally during this entire process will
  ensure that gameplay systems, UI elements, server endpoints, and
  blockchain interactions work cohesively.

  During this stage, reviewers and stakeholders will continue to provide
  their feedback in order to allow mid-development adjustments where
  necessary. This will ensure that the final build of the MPC remains in
  line with user expectations and the stated technical objectives.
\item
  \textbf{Implementation}

  The focus of the implementation phase will be the validation of the
  integrated system and preparation for evaluation as a working
  prototype. This will include extensive functional testing of gameplay
  systems, blockchain interactions, and the server-client communication
  pipeline. Of paramount importance in this phase will be the testing of
  the hybrid tokenomics model: primarily the balance between Perlas and
  Baku Coin, the stability of transactions, and how stabilization
  mechanisms behave under different simulated conditions.

  Consistent with RAD principles, any issues found in testing will be
  dealt with immediately, with changes being made in a rapid and focused
  manner to advance the prototype towards completion. This stage will
  also include the preparation of documentation, ensuring system
  readiness, and verifying that all major components function reliably
  under intended use conditions. The prototype, once polished, will be
  ready for review, showing not only the practical feasibility of the
  game system but also the effectiveness of the economic model proposed.
\end{enumerate}
