\section{Review of Related Literature, Studies or
System}\label{review-of-related-literature-studies-or-system}

\subsection{MMORPG Design: Engagement, Social Systems, and
Monetization}\label{mmorpg-design-engagement-social-systems-and-monetization}

The long-term success of decentralized MMORPGs depends on understanding
the psychological drivers of intrinsic motivation and building strong
social communities, moving beyond the failed, transaction-focused models
of early Web3 games. Sustainable player engagement relies on satisfying
three core psychological needs: Autonomy (a sense of choice), Competence
(the feeling of mastery), and Relatedness (social connection)
\autocite{Przybylski2024,Raza2024,Rigby2024}. Fulfilling these needs is
linked to better mental well-being \autocite{Raza2024} and forms the
foundation of genuine, intrinsic motivation \autocite{Rigby2024}.
Research confirms that players perceived fulfillment of autonomy and
competence is strongly tied to their sustained enjoyment of MMORPGs
\autocite{Taipalharju2020}.

This psychological framework directly shapes the Core Gameplay
Loop---the repeating cycle of Challenge, Actions, and Reward. However,
this loop is vulnerable to designs that over-emphasize hyper-efficiency
\autocite{Przybylski2024,Tangan2024}. An excessive focus on optimization
can create damaging Compulsion Loops, shifting player motivation from
intrinsic enjoyment to external habit \autocite{Przybylski2024}. This
negative shift is dramatically amplified in Play-to-Earn (P2E)
environments, where game efficiency is directly tied to real-world
monetary gain, thereby accelerating the transition toward habitual
rather than engaged play \autocite{Backe2023,Tangan2024}. Therefore, a
sustainable Play-to-Own (P2O) design must prioritize varied game
mechanics that support intrinsic growth, ensuring that the primary drive
for player engagement remains superior to minimal economic rewards
\autocite{Backe2023}.

Beyond individual motivation, social dynamics are crucial for retention.
High levels of sociability are consistently reported as a strong
predictor of both user satisfaction and long-term engagement in MMORPGs
\autocite{Kalyani2021}. Despite this, social systems that force
high-stakes interactions often lead to player toxicity and burnout
\autocite{Kalyani2021}. A strategic solution is to design modular,
opt-in ``social sinks''---game mechanics that reward positive,
low-stress contributions to the community \autocite{Sashi2021}.

Supporting this approach, experimental studies on social cooperation
show that rewards have a much stronger and more persistent effect on
encouraging cooperation than punishments \autocite{Seo2020}. This
finding mandates the design of non-monetary social sinks, such as
special status or unique titles, which leverage long-term psychological
fulfillment to reward community contributions \autocite{Sashi2021}.
While extrinsic incentives like points and rankings can boost short-term
motivation, long-term sustainability requires aligning these systems
with players' intrinsic motivations. Evidence for this exists in other
fields; for instance, gamification using non-extrinsic rewards has
successfully changed students' negative attitudes toward complex
financial topics \autocite{Sevidal2021}.

\subsection{Blockchain in Gaming: Tokenomics, Player-Driven Economies,
and P2E
vs.~P\&O}\label{blockchain-in-gaming-tokenomics-player-driven-economies-and-p2e-vs.-po}

The economic structure of Web3 gaming requires a fundamental shift from
speculative financial models toward sustainable systems built on
utility-based scarcity, secured by verifiable ownership and strong
governance. Blockchain technology enables this by granting players true,
verifiable ownership of unique digital assets (NFTs), providing a level
of control and permanence unattainable in traditional centralized
systems \autocite{Shen2024}. This capability has led to the emergence of
two distinct economic models.

The first, Play-to-Earn (P2E), relies on inflationary token emission, a
system where user engagement drops sharply when the tokens are no longer
profitable \autocite{Xie2024,Ali2023}. The real-world failure of this
model was demonstrated by the collapse of Axie Infinity, which caused
severe socioeconomic consequences for players in the Philippines. These
included financial losses of up to one million pesos and a forced return
to traditional gig work \autocite{Yu2021}. The impact was amplified
because many players understood their participation through high-risk
cultural models, such as viewing it as a digital version of sabong
(cockfighting) or a ``side hustle,'' which increased their exposure to
systemic risk \autocite{Tinio2023,Francisco2022}. Unlike models reliant
on token printing, the Play-to-Own (P\&O) paradigm builds a resilient
economy by tying asset value to intrinsic utility and proven scarcity.
This reduces the risk of inflation and ensures that player rewards are
linked to the game's enduring success, promoting greater economic
sustainability.

The foundation of these models is Tokenomics---the formal design of a
token's utility, supply, distribution, and circulation, which creates
the quantitative framework for economic stability
\autocite{DellaTorre2024}. A robust design requires algorithmic
regulatory controls for dynamic policy setting
\autocite{DellaTorre2024}. Furthermore, inflationary pressures can be
managed by applying Decentralized Finance (DeFi) principles, such as
using a contribution-reward model and Automated Market Maker (AMM)
mechanics to tie rewards directly to verifiable, value-added behavior
like burning NFTs \autocite{DellaTorre2023}. The critical link between
platform participation and financial health is confirmed by research
showing that a higher staking ratio for a governance token positively
predicts its market returns \autocite{Wang2024}.

NFTs are crucial for establishing digital property rights
\autocite{Shen2024} and enabling player-driven economies
\autocite{DellaTorre2023}. However, a significant challenge to their
potential is cross-platform interoperability. Moving assets between
different games is severely complicated by the risk of disrupting game
balance when an item is placed into a new mechanical context
\autocite{Yang2024}. The lack of standardized protocols for how these
assets are represented across platforms remains a major technical hurdle
\autocite{Yang2024}.

\subsection{Filipino Culture in Digital Media: Identity, Narrative, and
the Social
Sink}\label{filipino-culture-in-digital-media-identity-narrative-and-the-social-sink}

Integrating Filipino cultural informatics is vital for forging authentic
game narratives, affirming national identity, and building socially
resilient community structures. The dominance of the Western video game
industry often pushes Filipino developers toward outsourced work,
hindering the creation of original intellectual property defined by
local narratives \autocite{Tangan2022}. Fortunately, Philippine
mythology, folklore, and history offer an extensive source for narrative
and design material, often containing environmental consciousness and
moral wisdom \autocite{Tangan2022,Mangaldan2023,KaganFolktales2024}.

Academic case studies validate this approach. The RPG Anito: Defend a
Land Enraged, which uses allegorical settings and features the
tikbalang, demonstrated the successful adaptation of local narratives
into Western genres \autocite{Tangan2022,Tangan2023}. The development of
Anito: Battle of the Gods further confirmed this, successfully merging
entertainment and education (``edutainment'') and resulting in
resounding player satisfaction \autocite{Garcia2023,Zainuddin2020}.
However, designers must critically address the risk of
self-exoticization---treating Philippine culture as superficial ``window
dressing'' to market copies of foreign genres \autocite{Tangan2023}.
True cultural representation requires moving beyond a narrative ``skin''
to fundamentally reimagine Filipinoness within the game mechanics
themselves \autocite{Tangan2023,Tangan2024}.

This is crucial because video games are established as a medium with
significant potential to enhance players' cultural awareness and develop
socio-cultural literacy regarding diverse geopolitical spaces
\autocite{Shliakhovchuk2020}. The strategy of glocalization---adapting
global formats to local culture---is achieved through specific
representations in visual, characterization, and socio-linguistic
choices, ensuring resonance with local audiences
\autocite{Lopez2024,Cabañes2023}.

The concept of the ``social sink'' offers a powerful avenue for this
deep cultural integration. By modeling guild or community mechanics on
existing Filipino social tools, such as the communal spirit of bayanihan
(collective effort), designers can create culturally resilient social
sinks \autocite{Tinio2023,SanLuis2024}. Studies on Filipino gamers
confirm that virtual subcultures often reflect and adapt real-life
customs and traditions, relying on the Filipino language to foster a
``freer'' social space for identity expression and community bonding
\autocite{SanLuis2024,Plaridel2024}. This structural approach promotes
collaboration and shared responsibility, effectively minimizing the
high-stakes, stressful social dynamics observed in many traditional
MMORPGs \autocite{Jäger2021}.

\subsection{Technical System Architecture: Scalability, Services, and
Smart
Contracts}\label{technical-system-architecture-scalability-services-and-smart-contracts}

The foundation for a Web3 MMORPG mandates a hybrid technical
architecture designed for real-time performance, cryptographic
integrity, and efficient asset management. A scalable MMORPG
architecture requires a microservice design utilizing container
orchestration \autocite{Oyeniran2024}. Kubernetes coupled with Agones is
the essential solution for managing the server lifecycle, enabling
rapid, automated spin-up and horizontal auto-scaling
\autocite{Genieee2024}. Latency mitigation is essential, requiring
Machine Learning (ML) prediction to forecast network latency and
tolerate spikes \autocite{PolitecnicodiTorino2025}.

The application service layer should be built on ASP.NET Core due to its
high-performance capabilities. Using SignalR, ASP.NET applications
implement real-time functionality by enabling server-side code to push
content to clients, utilizing WebSockets and integrating seamlessly with
core ASP.NET features like authentication \autocite{Ably2024}. This
service layer acts as the authoritative intermediary, managing the
persistent economic state and synchronizing the low-latency game
simulation with the secure blockchain ledger \autocite{Genieee2024}.

For asset management, ERC-1155 is optimal for MMORPG inventory due to
its multi-token framework, which manages both fungible and non-fungible
assets in a single contract, enabling efficient batch transfers.
Security best practices require strict access controls and
multi-signature approvals for critical operations. ERC-721 is reserved
for truly unique assets, such as major land plots \autocite{Shen2024}.
For genuine Player-to-Own integrity, best practices mandate using
content-addressed URIs (e.g., IPFS CIDs) for linking metadata to the
token, ensuring the metadata defining asset utility is immutable and
persistent \autocite{Shen2024}.

Game Theory is the analytical framework for engineering decentralized
stability \autocite{Tandon2024,AbouSafaqa2024}. The system must reach a
stable Nash Equilibrium, where honest participation is the dominant
rational strategy \autocite{Tandon2024}. Achieving the desired stability
level of last-iterate convergence requires mathematically enforced
mechanisms, such as making the game ``strongly monotone''
\autocite{Liu2025}.

System integrity is constantly threatened by Sybil attacks, where a
single entity uses multiple identities to gain control in DAOs or
manipulate virtual economies
\autocite{Tandon2024,Victor2020,Humanode2024}. AI/ML tools are essential
for real-time fraud detection by analyzing on-chain data
\autocite{Hassan2025,CoherentMarketInsights2025}. The deployment of
Explainable AI (XAI) is critical, as it makes the AI's decision-making
logic transparent, increasing stakeholder confidence and aiding
regulatory compliance \autocite{Boopathi2024}. Autonomous AI agents are
projected to manage complex virtual economies as `flexible capital,'
potentially implementing dynamic pricing
\autocite{Sager2025,Saharan2025}. Prudent risk mitigation suggests
limiting the value of assets these agents can control
\autocite{Kumari2021}. The Philippines' integration of ABCD technologies
faces challenges due to infrastructure constraints and skill gaps
\autocite{Rodriguez2024}.

\subsection{Synthesis and Identification of the Research
Gap}\label{synthesis-and-identification-of-the-research-gap}

The literature establishes that the first generation of Play-to-Earn
(P2E) Web3 games, while pioneering digital asset ownership, suffered
from critically flawed economic models characterized by inflationary
token emission and a fundamental imbalance between resource faucets and
sinks, leading to inevitable hyperinflation and economic collapse
\autocite{Xie2024,Ali2023}. This technical failure was compounded by a
sociological flaw: an over-reliance on extrinsic, profit-seeking
motivation, which cultivated a user base of speculators rather than an
invested community, making these economies highly vulnerable to market
volatility \autocite{Backe2023,Tangan2024}. In response, the paradigm is
shifting towards Play-to-Own (P\&O), which emphasizes utility-based
scarcity and verifiable ownership as foundations for a more resilient
economy \autocite{Shen2024,DellaTorre2023}.

Concurrently, decades of research in player motivation confirm that
long-term engagement is not sustained by extrinsic rewards alone but is
fundamentally driven by the fulfillment of intrinsic psychological needs
for autonomy, competence, and relatedness, as outlined by
Self-Determination Theory \autocite{Przybylski2024,Rigby2024}. This is
particularly relevant to MMORPGs, where social dynamics and a sense of
community are consistently reported as the strongest predictors of user
satisfaction and retention \autocite{Kalyani2021}. The strategic design
of ``social sinks''---game mechanics that reward positive, low-stress
community contributions---is posited as a method to foster this
intrinsic motivation and relatedness \autocite{Sashi2021,Seo2020}.

Furthermore, the integration of rich cultural narratives, such as those
found in Filipino mythology, is shown to be a powerful tool for forging
authentic game worlds and affirming identity, moving beyond superficial
``window dressing'' to create deeply resonant experiences
\autocite{Tangan2022,Tangan2023}. From a technical perspective,
achieving a stable Web3 MMORPG requires a sophisticated hybrid
architecture that leverages microservices and container orchestration
for scalability \autocite{Oyeniran2024,Genieee2024}, while smart
contracts must incorporate advanced mechanisms like Automated Market
Makers (AMMs) and anti-exploit systems to ensure economic stability and
security \autocite{DellaTorre2024,Tandon2024}.

However, a critical research gap persists at the convergence of these
domains. While the individual components---sustainable tokenomics,
intrinsic motivation drivers, cultural narrative integration, and
scalable hybrid architecture---are well-discussed in isolation, there is
a lack of a unified framework that integrates them into a cohesive
design. Specifically, there is an absence of a proven model that
operationalizes a culturally-grounded ``social sink'' as a core economic
and engagement mechanic within a programmatically stabilized P\&O
tokenomic system. Existing literature either addresses economic
stabilization in a cultural vacuum or discusses cultural engagement
without a concrete link to blockchain economic sustainability.

Therefore, this study seeks to address this gap by proposing and
prototyping a novel hybrid model for a Web3 MMORPG. The research will
investigate how a tokenomic system, stabilized through an AMM, adaptive
throttles, and a progressive anti-exploit framework, can be
synergistically combined with a gameplay loop and social systems deeply
embedded in Filipino mythology. The central hypothesis is that this
integration will create a more resilient and sustainable virtual economy
by technically mitigating inflationary pressures while sociologically
cultivating a player base motivated by intrinsic engagement and cultural
identity, rather than short-term financial extraction.

\begin{figure}[H]
\centering
\pandocbounded{\includegraphics[width=\linewidth]{assets/rrl/rrl-overview.pdf}}
\caption{Overview of Review of Related Literature Diagram}
\end{figure}

\begin{figure}[ht]
\centering
\begin{minipage}{0.48\textwidth}
    \centering
    \includegraphics[width=\linewidth]{assets/rrl/Block-chain-in-gamin.pdf}
    \caption{Blockchain in Gaming}
\end{minipage}
\hfill
\begin{minipage}{0.48\textwidth}
    \centering
    \includegraphics[width=\linewidth]{assets/rrl/Filipino-Culture-in-Digital-Media.pdf}
    \caption{Filipino Culture in Digital Media}
\end{minipage}
\end{figure}

\begin{figure}[ht]
\centering
\begin{minipage}{0.50\textwidth}
    \centering
    \includegraphics[width=\linewidth]{assets/rrl/mmo-rpg-design.pdf}
    \caption{MMORPG Design}
\end{minipage}
\hfill
\begin{minipage}{0.48\textwidth}
    \centering
    \includegraphics[width=\linewidth]{assets/rrl/Technical-System-Architecture.pdf}
    \caption{Technical System Architecture}
\end{minipage}
\end{figure}
