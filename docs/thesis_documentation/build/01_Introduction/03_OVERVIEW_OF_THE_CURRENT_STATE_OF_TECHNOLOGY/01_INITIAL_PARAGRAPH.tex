\section{Overview of the current state of the
technology}\label{overview-of-the-current-state-of-the-technology}

The current landscape of blockchain-based MMORPGs is defined by a
fundamental conflict between the technical requirements of massive
multiplayer games and the inherent properties of decentralized networks.
Consequently, the industry's initial approach, embodied by the
Play-to-Earn (P2E) model, has proven critically flawed, leading to
unsustainable economies and poor player experiences.

Modern MMORPGs rely on a server-authoritative, client-server
architecture to manage real-time state for thousands of concurrent
players. This model demands low-latency communication and high-frequency
updates to create a seamless, responsive world. In contrast, blockchain
networks prioritize security and verifiability through decentralized
consensus, a process that introduces significant latency and transaction
costs. As a result, early Web3 games that attempted to bridge this gap
by running core logic on-chain were often slow, expensive, and
technically limited, ultimately failing to compete with traditional
titles.

Furthermore, the first generation of P2E games established an economic
paradigm centered on inflationary token emission. Critically, these
models lacked robust economic ``sinks'' to balance the prolific resource
``faucets,'' which inevitably led to hyperinflationary death spirals. As
token rewards lost value, engagement became solely tied to speculative
profit, thereby attracting a user base of mercenary speculators rather
than dedicated players. The subsequent economic collapses, as seen in
cases like Axie Infinity, demonstrated the tangible risks of these
unstable virtual economies and their profound real-world consequences
for dependent players.

Moreover, this economic flaw is compounded by a profound motivational
misalignment. Decades of research in player psychology, grounded in
Self-Determination Theory, confirm that long-term engagement in virtual
worlds is driven by intrinsic motivation---the fulfillment of autonomy,
competence, and relatedness. However, the P2E model, by over-relying on
extrinsic financial rewards, actively undermines these psychological
drivers. It effectively replaces the joy of play with the compulsion of
work, thereby fostering a speculative mindset that is highly vulnerable
to market volatility and ultimately erodes the foundation of a stable,
long-term community.

Finally, a core promise of Web3 gaming---asset
interoperability---remains a significant challenge. While the vision is
to use NFTs across different gaming platforms, this goal faces major
technical and design hurdles. Specifically, there is a lack of
standardized protocols for representing assets across different game
worlds. In addition, importing a powerful asset from one game into
another severely risks disrupting the mechanical context and balance of
the recipient game, a problem that remains largely unsolved.

In conclusion, the current state of Web3 MMORPGs is hampered by a core
architectural conflict between performance and decentralization,
economically unsustainable token models that promote speculation over
play, a fundamental misalignment with the psychological drivers of
long-term engagement, and significant unresolved technical barriers to
achieving true asset interoperability.
