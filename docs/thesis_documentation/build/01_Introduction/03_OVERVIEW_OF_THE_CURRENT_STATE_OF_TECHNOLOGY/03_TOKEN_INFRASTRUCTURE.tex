\subsection{Blockchain Infrastructure in
Web3}\label{blockchain-infrastructure-in-web3}

Blockchain technology has become a fundamental component of contemporary
Web3 gaming, addressing several limitations that affected earlier
implementations. Initial Play-to-Earn platforms primarily operated on
the Ethereum network, which frequently experienced congestion and
substantially high transaction costs. These factors created significant
financial barriers for participants {[}@Yu2022{]}. In response, the
industry shifted toward scaling solutions such as the Polygon network,
which maintains compatibility with Ethereum's ecosystem while offering
improved transaction speeds and reduced fees, establishing itself as a
preferred platform for gaming applications {[}@Polygon2023{]}.

Middleware solutions represent another critical component, serving as
intermediary software that facilitates communication between game
clients and blockchain networks. These systems manage essential
functions including data retrieval and transaction processing, ensuring
responsive gameplay despite inherent blockchain latency
{[}@Moralis2023{]}. The integration of non-custodial wallet systems
enables secure management of digital assets, granting users true
ownership of in-game items. Collectively, these technological
elements---scalable networks, middleware interfaces, and wallet
systems---establish a robust foundation for developing functional and
decentralized gaming experiences.

This project utilizes this advanced infrastructure to resolve the
ongoing challenge of optimizing both economic efficiency and security.
We have developed a hybrid architectural model that employs conventional
game servers to manage high-frequency, non-critical gameplay operations.
This methodology significantly reduces transaction costs while
maintaining performance standards. The blockchain component is
specifically allocated for crucial operations, including the creation
and exchange of NFT-based game assets and significant currency
conversions. Through this strategic distribution of computational tasks,
this framework preserves the security advantages and authentic ownership
capabilities of decentralized systems for valuable assets, while
circumventing the excessive operational expenses that hindered previous
Web3 gaming implementations.

\subsubsection{Smart Contract}\label{smart-contract}

Smart contracts function as autonomous programs that govern economic
rules and digital ownership within Web3 gaming environments. Industry
standards such as ERC-20 {[}@Vogelsteller2015erc20{]} for fungible
tokens and ERC-721 {[}@Entriken2018erc721{]} for non-fungible assets
provide the foundational framework for most implementations, ensuring
interoperability across platforms and wallets.

Contemporary smart contract designs incorporate advanced mechanisms to
maintain economic stability. Automated Market Makers (AMMs), adapted
from decentralized finance protocols, facilitate price discovery and
liquidity provision through algorithmic functions
{[}@Adams2021Uniswap{]}. Additionally, programmable monetary policies
enable automatic adjustment of token creation and removal rates, serving
as countermeasures against inflationary pressures
{[}@Gkillas2022Game{]}.

Security measures have evolved to address exploitation risks, with
modern contracts implementing constraints such as daily conversion
ceilings and minimum level requirements. These mechanisms deter purely
extractive behavior by preventing immediate token conversion for new
accounts, thereby encouraging genuine gameplay participation rather than
financial speculation.

This project implements customized ERC-20 and ERC-721 smart contracts
featuring multiple economic stabilization mechanisms. The architecture
employs a dual-token system comprising Perlas as the in-game currency
and Baku Coin as the blockchain-based asset, integrated with automated
balancing protocols. The contract design incorporates adaptive minting
rates that respond to market conditions and integrated economic sinks
that systematically remove tokens from circulation. Furthermore, the
implementation includes anti-exploit protections such as conversion
restrictions and level-based access limits to prevent economic
manipulation by participants seeking solely to generate revenue.
