\chapter{Introduction}\label{introduction}

The rise of Massively Multiplayer Online Role-Playing Games (MMORPGs)
has transformed digital entertainment by offering persistent worlds,
strong social interaction, and complex, player-driven economies. With
the emergence of Web3 innovations, developers envisioned a new evolution
for this genre---one where players could enjoy true digital asset
ownership, decentralized governance, and earning potential through
blockchain technology. This vision ignited global interest, leading many
to believe that blockchain-powered MMORPGs would define the next era of
the ``Play-to-Earn'' (P2E) movement.

Early Web3 titles such as Axie Infinity and MIR4 showcased this
potential by enabling players to convert their in-game progress into
real-world income. However, these pioneering systems ultimately
collapsed due to major technical and social shortcomings. Economically,
their models suffered from severe imbalance---token ``faucets,'' or
sources of token creation, far exceeded the available ``sinks,'' or
mechanisms that removed tokens from circulation. This structural flaw
led to hyperinflation and collapsed token value once user growth slowed.
Sociologically, these systems attracted players driven primarily by
extrinsic, profit-oriented motivations. This ``speculator mindset''
weakened community cohesion, destabilized the game economy, and limited
long-term engagement.

These failures reveal a significant gap in current Web3 game design: the
absence of a sustainable framework that combines economic stability with
intrinsic player motivation. To address this gap, this research
introduces \textbf{Echoes of Bathala Online}, an MMORPG prototype that
blends blockchain technology with cultural immersion inspired by
Filipino mythology. The project proposes a novel hybrid stabilization
framework featuring a multi-layered economic system---consisting of an
Automated Market Maker (AMM), macroeconomic throttling tools, and
progressive anti-exploit mechanisms---to ensure programmatic economic
stability. Complementing this is a ``social sink'' grounded in
Self-Determination Theory, using narrative depth, cultural
worldbuilding, and meaningful social systems to cultivate intrinsic
motivation and long-term community resilience.

This study aims to design, develop, and evaluate this prototype to
demonstrate the feasibility of integrating stable blockchain-driven
mechanics with culturally meaningful gameplay. Specifically, it seeks to
architect a dual-token economy connecting the in-game \textbf{Perlas}
currency with the blockchain-based \textbf{Baku Coin}, implement core
gameplay systems that support this economic architecture, and showcase a
secure bridge between a traditional game server and decentralized
blockchain components. Through this proof-of-concept, \textbf{Echoes of
Bathala Online} aspires to provide a sustainable blueprint for the next
generation of Web3 MMORPGs---one that is economically sound, culturally
grounded, and deeply engaging.
